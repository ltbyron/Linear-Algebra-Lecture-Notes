\documentclass[hyperref,]{ctexart}
\usepackage{lmodern}
\usepackage{amssymb,amsmath}
\usepackage{ifxetex,ifluatex}
\usepackage{fixltx2e} % provides \textsubscript
\ifnum 0\ifxetex 1\fi\ifluatex 1\fi=0 % if pdftex
  \usepackage[T1]{fontenc}
  \usepackage[utf8]{inputenc}
\else % if luatex or xelatex
  \ifxetex
    \usepackage{xltxtra,xunicode}
  \else
    \usepackage{fontspec}
  \fi
  \defaultfontfeatures{Mapping=tex-text,Scale=MatchLowercase}
  \newcommand{\euro}{€}
\fi
% use upquote if available, for straight quotes in verbatim environments
\IfFileExists{upquote.sty}{\usepackage{upquote}}{}
% use microtype if available
\IfFileExists{microtype.sty}{%
\usepackage{microtype}
\UseMicrotypeSet[protrusion]{basicmath} % disable protrusion for tt fonts
}{}
\ifxetex
  \usepackage[setpagesize=false, % page size defined by xetex
              unicode=false, % unicode breaks when used with xetex
              xetex]{hyperref}
\else
  \usepackage[unicode=true]{hyperref}
\fi
\usepackage[usenames,dvipsnames]{color}
\hypersetup{breaklinks=true,
            bookmarks=true,
            pdfauthor={李弢},
            pdftitle={线性代数第六次习题课},
            colorlinks=true,
            citecolor=blue,
            urlcolor=blue,
            linkcolor=magenta,
            pdfborder={0 0 0}}
\urlstyle{same}  % don't use monospace font for urls
\setlength{\emergencystretch}{3em}  % prevent overfull lines
\providecommand{\tightlist}{%
  \setlength{\itemsep}{0pt}\setlength{\parskip}{0pt}}
\setcounter{secnumdepth}{5}

\title{线性代数第六次习题课}
\author{李弢}
\date{2019年12月12日}

% Redefines (sub)paragraphs to behave more like sections
\ifx\paragraph\undefined\else
\let\oldparagraph\paragraph
\renewcommand{\paragraph}[1]{\oldparagraph{#1}\mbox{}}
\fi
\ifx\subparagraph\undefined\else
\let\oldsubparagraph\subparagraph
\renewcommand{\subparagraph}[1]{\oldsubparagraph{#1}\mbox{}}
\fi

\begin{document}
\maketitle

{
\setcounter{tocdepth}{2}
\tableofcontents
}
\def\vec{\overrightarrow} \def\bfF{\mathbf{F}} \def\calP{\mathcal{P}}
\def\calC{\mathcal{C}} \def\calL{\mathcal{L}} \def\calM{\mathcal{M}}
\def\R{\mathbb{R}} \def\C{\mathbb{C}} \def\N{\mathbb{N}}
\def\Z{\mathbb{Z}} \def\span{\mathrm{span}\,} \def\dim{\mathrm{dim}\,}
\def\Null{\mathrm{null}\,} \def\range{\mathrm{range}\,}
\def\rank{\mathrm{rank}\,} \def\proj{\mathrm{Proj}\,}
\def\st{\mathrm{s.t.}} \def\d{\mathrm{d}\,}

\section{线性泛函与伴随}\label{ux7ebfux6027ux6cdbux51fdux4e0eux4f34ux968f}

\noindent{} 1. 取定向量\(v\in V\),
把\(T\in\calL(V,\bfF)\)定义成\(Tu = \langle u,v\rangle\).
求\(T^*a,\forall a\in\bfF\).

\smallskip

\noindent{}
解:\(\langle u,T^*a\rangle_V= \langle Tu,a\rangle_{\bfF} =\overline{a}\langle u,v\rangle_V = \langle u,av\rangle_V\).
因此\(T^*a = av\).

\bigskip

\noindent{} 2.
设\(n\)为整数,把\(T\in\calL(\bfF^n)\)定义为\(T(z_1,z_2,\cdots,z_n)=(0,z_1,\cdots,z_{n-1})\),
求\(T^*(z_1,z_2,\cdots,z_n)\).

\smallskip

\noindent{} 解:
\[\langle T(z_1,z_2,\cdots,z_n),(w_1,w_2,\cdots,w_n)\rangle = \langle(z_1,z_2,\cdots,z_n),T^*(w_1,w_2,\cdots,w_n)\rangle\]
由\(T\)的定义可知
\[\langle T(z_1,z_2,\cdots,z_n),(w_1,w_2,\cdots,w_n)\rangle = w_2z_1+w_3z_2+\cdots+w_nz_{n-1},\]
因此\(T^*(w_1,w_2,\cdots,w_n)=(w_2,w_3,\cdots,w_n,0)\).

\smallskip

\noindent{} Remark:
对于\(\bfF^n\),可以定义很多不同的内积,但一般默认情况下我们就认为是标准内积,也就是对应坐标相乘再相加的形式。

\bigskip

\noindent{} 3. 设\(T\in\calL(V)\)且\(\lambda\in\bfF\),
证明\(\lambda\)是\(T\)的特征值当且仅当\(\overline{\lambda}\)是\(T^*\)的特征值。

\smallskip

\noindent{}
解:\(\lambda\)是\(T\)的特征值,即\(\Null(T-\lambda I)\ne\{0\}\).
此外,注意到\(T^*-\overline{\lambda}I\)就是\(T-\lambda I\)的伴随算子,
由Fredholm选择定理,\(\range(T^*-\overline{\lambda}I) =\Null(T-\lambda I)\),
因此\(\dim \range(T^*-\overline{\lambda}I) = \dim V- \dim \Null(T-\lambda I)<\dim V\),
那么\(\dim \Null(T^*-\overline{\lambda}I) = \dim V-\dim \range(T^*-\overline{\lambda}I)>0\),
这就说明了\(\Null(T^*-\overline{\lambda} I)\ne\{0\}\),
即\(\overline{\lambda}\)是\(T^*\)的特征值。另一方面,由于\(T^{**}=T,\overline{\overline{\lambda}}=\lambda\),
可知结论也成立。

\bigskip

\noindent{} 4. 设\(T\in\calL(V)\),
并且\(U\)是\(V\)的子空间,证明\(U\)在\(T\)下是不变的当且仅当\(U^\perp\)在\(T^*\)下是不变的。

\smallskip

\noindent{} 解:若\(U\)在\(T\)下不变,则对任意的\(u\in U, w\in W\),
有\(Tu\in U\),所以\(\langle Tu,w\rangle=0\),
这又推出\(\langle u, T^*w\rangle=0\), 即说明\(T^*w\in U^\perp\).
所以\(U^\perp\)是\(T^*\)的不变子空间。另一方面,\((U^\perp)^\perp=U,T^{**}=T\),
故若\(U^\perp\)在\(T^*\)下不变也可以推出\(U\)在\(T\)下不变。

\bigskip

\noindent{} 5. 设\(T\in\calL(V,W)\), 证明:

\begin{itemize}
\tightlist
\item
  (a). \(T\)是单的当且仅当\(T^*\)是满的;
\item
  (b). \(T\)是满的当且仅当\(T^*\)是单的。
\end{itemize}

\smallskip

\noindent{} 解:由于\(T^{**}=T\), 我们只需证明\(T\)是单的\(\Rightarrow\)
\(T^*\)是满的,\(T\)是满的\(\Rightarrow\) \(T^*\)是单的。

\begin{itemize}
\tightlist
\item
  (a). 设\(T\)是单的,即\(\Null T=\{0\}\).
  由Fredholm选择定理可知\(\range T^*=(\Null T)^{\perp}=V\),
  故\(T^*\)是满的。
\item
  (b). 设\(T\)是满的,即\(\range T=W\).
  由Fredholm选择定理可知\(\Null T^* = (\range T)^{\perp}=\{0\}\),
  故\(T^*\)是单的。
\end{itemize}

\smallskip

\noindent{} Remark:
我们已经知道在一组规范正交基下,\(T\)与\(T^*\)的矩阵互为共轭转置。在矩阵的意义下,\(T\)是满的等价于\(\calM(T)\)列满秩,\(T\)是单的等价于\(\calM(T)\)行满秩,我们又知道\(\calM(T^*)=\calM(T)^*\),
因此显然\(\calM(T)\)列满秩(\(T\)满)等价于\(\calM(T^*)\)行满秩(\(T^*\)单),\(\calM(T)\)行满秩(\(T\)单)等价于\(\calM(T^*)\)列满秩(\(T^*\)满)。

\bigskip

\noindent{} 6. 证明对每个\(T\in\calL(V,W)\)都有
\[\dim \Null T^*=\dim \Null T+\dim W-\dim V.\]
并且\(\dim\range T=\dim\range T^*\).

\smallskip

\noindent{} 解: \[\begin{aligned}
\dim\Null T^* &= \dim W-\dim \range T^* \\
&= \dim W-\dim (\Null T)^\perp\\
&= \dim W-(\dim V-\dim\Null T)\\
&= \dim \Null T+\dim W-\dim V.
\end{aligned}\]
又由于\(\dim \range T=\dim V-\dim \Null T, \dim \range T^*=\dim W-\dim \Null T^*\),
易知\(\dim\range T=\dim\range T^*\).

\smallskip

\noindent{} Remark: 在规范正交基下,\(\calM(T^*)=\calM(T)^*\),
而\(\dim \range T\)就是\(\calM(T)\)的列秩,\(\dim\range T^*\)就是\(\calM(T^*)\)的列秩,实际上就是\(\calM(T)\)的行秩。这个结果实际上就等价于一个矩阵的列秩等于行秩。

\bigskip

\noindent{} 7.
设\(A\)为\(m\times n\)的实矩阵。证明\(A\)的所有列(在\(\R^m\)中)张成的子空间的维数等于\(A\)的所有行(在\(\R^n\)中)张成的子空间的维数。

\smallskip

\noindent{} 解:考虑一个线性映射\(T:\R^n\rightarrow\R^m,T(x)=Ax\).
考虑\(\R^m\)和\(\R^n\)的标准基,在这组基下\(\calM(T^*)=A^{\mathrm{T}}\).
那么\(A\)的所有列(在\(\R^m\)中)张成的子空间的维数等于\(\dim\range T\),\(A\)的所有行(在\(\R^n\)中)张成的子空间的维数等于\(\dim\range T^*\).
由上题可知\(\dim\range T=\dim\range T^*\), 故证得。

\section{自伴算子与正规算子}\label{ux81eaux4f34ux7b97ux5b50ux4e0eux6b63ux89c4ux7b97ux5b50}

\noindent{} 1. 按照下面的定义,\(\calP_2(\R)\)是内积空间,
\[\langle p,q\rangle =\int_0^1 p(x)q(x)\d x.\]
定义\(T\in\calL(\calP_2(\R))\)使得\(T(a_0+a_1x+a_2x^2)=a_1x\).

\begin{itemize}
\tightlist
\item
  (a). 证明\(T\)不是自伴的。
\item
  (b). \(T\)关于\((1,x,x^2)\)的矩阵是 \[
  \begin{pmatrix}
  0&0&0\\
  0&1&0\\
  0&0&0.
  \end{pmatrix}
  \] 虽然\(T\)不是自伴的,但是这个矩阵却和它的共轭转置相等。
  解释为什么这并不矛盾。
\end{itemize}

\smallskip

\noindent{} 解:

\begin{itemize}
\item
  (a).
  假设\(T\)是自伴的,那么\(\langle Tp,q\rangle = \langle p,Tq\rangle\)
  对任何\(p,q\in\calP_2(\R)\)均成立。取\(p(x)=x^2,q(x)=x\), 那么
  \[\langle Tp,q\rangle = \langle 0,x\rangle=0.\]
  \[\langle p,Tq\rangle = \langle x^2,x\rangle=\int_0^1x^3\d x=\frac14\ne0.\]
  因此,\(T\)不可能是自伴的。
\item
  (b).
  之前我们是说在一组规范正交基下,伴随算子的矩阵等于原算子矩阵的共轭转置,因而自伴算子的矩阵等于它的共轭转置。但这题里考虑的基\((1,x,x^2)\)在这个内积的定义下不是规范正交基,故虽然这个矩阵是对称的,也无法说明算子是自伴的。
\end{itemize}

\bigskip

\noindent{} 2.

\begin{itemize}
\tightlist
\item
  (a).
  证明:若\(V\)是实内积空间,则\(V\)上的自伴算子之集是\(\calL(V)\)的子空间。
\item
  (b).
  证明:若\(V\)是复内积空间,则\(V\)上的自伴算子之集不是\(\calL(V)\)的子空间。
\end{itemize}

\smallskip

\noindent{} 解:

\begin{itemize}
\item
  (a). 任取两个实内积空间上的自伴算子\(T,S\),
  我们来说明\(\forall \lambda,\mu\in\R,\lambda T+\mu S\)仍是自伴算子。
  \[\langle (\lambda T+\mu S)u,v\rangle = \lambda \langle Tu,v\rangle+\mu\langle Su,v\rangle=\langle u,\lambda T^*v\rangle+\langle u,\mu S^*v\rangle=\langle u,(\lambda T+\mu S)v\rangle.\]
  因此实内积空间\(V\)上的自伴算子之集是\(\calL(V)\)的子空间。
\item
  (b).
  设\(T\)是复内积空间\(V\)上的一个自伴算子,我们来说明\(\lambda T\)有可能不是自伴的。
  \[\langle \lambda Tu,v\rangle = \lambda\langle T,u\rangle=\lambda\langle u,Tv\rangle=\langle u,\overline{\lambda}Tv\rangle.\]
  这说明\((\lambda T)^* = \overline{\lambda} T\),
  当\(\Im\lambda\ne0\)且\(T\ne 0\)时\((\lambda T)^*\ne\lambda T\),
  这就说明了复内积空间\(V\)上的自伴算子之集不是\(\calL(V)\)的子空间。
\end{itemize}

\bigskip

\noindent{} 3. 设\(P\in\calL(V)\)使得\(P^2=P\).
证明\(P\)是正交投影当且仅当\(P\)是自伴的。

\smallskip

\noindent{} 解:由\(P^2=P\)可知存在\(U\oplus W=V,P=P_{U,W}\).

\begin{itemize}
\item
  (\(\Rightarrow\):)
  \(P\)是正交投影,即是说存在\(V\)的子空间\(U\),\(P=P_{U,U^\perp}\).
  对任意\(v_1,v_2\in V\), 我们将他们分解为
  \[v_1 = u_1+w_1,v_2=u_2+w_2,u_1\in U,u_2\in U,w_1\in U^\perp, w_2\in U^\perp.\]
  那么
  \[\langle Pv_1,v_2\rangle = \langle u_1,u_2+w_2\rangle = \langle u_1,u_2\rangle =\langle u_1+w_1,u_2\rangle=\langle v_1,Pv_2\rangle.\]
  这就说明了\(P\)是自伴的。
\item
  (\(\Leftarrow:\))
  由Fredholm选择定理可知,\(\range P^*=(\Null P)^\perp\).
  而\(P\)又是自伴的,所以\(\range P=\range P^*=(\Null P)^\perp\).
  再结合\(P^2=P\), 由上一章一道课后题的结论可知,\(P\)是正交投影。
\end{itemize}

\smallskip

\noindent{} Remark:
在无限维空间中,有时候很难说清把一个空间分为两个子空间或者正交子空间的直和。相反,\(P^2=P\)就是投影算子的一般定义,\(P^2=P\)且\(P\)自伴就是正交投影算子的一般定义。

\bigskip

\noindent{} 4. 证明:若\(\dim V\ge 2\),
则\(V\)上的正规算子之集不是\(\calL(V)\)的子空间。

\smallskip

\noindent{}
解:设\(V\)是一个维数大于等于2的线性空间,取一组规范正交基\((e_1,e_2,\cdots,e_n)\).
定义\(S,T\in\calL(V)\):
\[S(a_1e_1+a_2e_2+\cdots+a_ne_n) = a_2e_1-a_1e_2,\]
\[T(a_1e_1+a_2e_2+\cdots+a_ne_n) = a_2e_1+a_1e_2.\]
我们先验证\(S\)和\(T\)都是正规算子,先来求\(S^*\):
\[\langle a_1e_1+\cdots+a_ne_n,S^*(b_1e_1+\cdots+b_ne_n)\rangle =\langle a_2e_1-a_1e_2,b_1e_1+\cdots+b_ne_n\rangle=a_2b_1-a_1b_2.\]
因此\(S^*(b_1e_1+\cdots+b_ne_n)=-b_2e_1+b_1e_2\), 那么
\[SS^*(a_1e_1+a_2e_2+\cdots+a_ne_n)=S(-a_2e_1+a_1e_2)= a_1e_1+a_2e_2.\]
\[S^*S(a_1e_1+a_2e_2+\cdots+a_ne_n)=S^*(a_2e_1-a_1e_2)= a_1e_1+a_2e_2=SS^*.\]
这说明\(S\)是正规算子。类似地可以验证\(T\)也是正规算子。而\((S+T)(a_1e_1+\cdots+a_ne_n) = 2a_2e_1\),
\((S+T)^*(a_1e_1+\cdots+a_ne_n) =2a_1e_2\), 因此
\[(S+T)(S+T)^*(a_1e_1+a_2e_2+\cdots+a_ne_n)=4a_1e_1,\]
\[(S+T)^*(S+T)(a_1e_1+a_2e_2+\cdots+a_ne_n)=4a_2e_2,\]
这说明\(S+T\)不是正规算子。

\bigskip

\noindent{} 5. 证明:
若\(T\in\calL(V)\)是正规的,则\(\range T=\range T^*\).

\smallskip

\noindent{} 解:我们先说明\(\Null T=\Null T^*\).
由于\(T\)是正规算子,\(\|Tv\|^2=\langle Tv,Tv\rangle = \langle v,T^*Tv\rangle = \langle v,TT^*v\rangle=\langle T^*v,T^*v\rangle =\|T^*v\|^2\).
因此,若\(v\in\Null T\), \(\|T^*v\| =\|Tv\|=0\),
由范数的定性可知\(T^*v=0\), 即\(v\in\Null T^*\). 由于\(T^{**}=T\),
反之亦然,故\(\Null T=\Null T^*\).
而由Fredholm选择定理,\(\range T=(\Null T^*)^\perp,\range T^*=(\Null T)^\perp\),
由正交补空间的唯一性可知\(\range T =\range T^*\).

\bigskip

\noindent{} 6. 若\(T\in\calL(V)\)是正规的,则对每个正整数\(k\)都有
\[\Null T^k=\Null T,\range T^k=\range T.\]

\smallskip

\noindent{} 解:我们先证明\(\Null T^k=\Null T\).
首先\(\Null T\subset \Null T^k\)是显然的,故只需证明\(\forall v\in\Null T^k\)可推出\(v\in\Null T\).
由于\(T\)是正规算子,\(\|Tu\|=\|T^*u\|\)对任意\(u\in V\)成立。特别地,取\(u=T^{k-1}v\)可知,\(\|T^*T^{k-1}v\|=\|T^k\|=0\),
即\(\T^*T^{k-1}v=0\). 考虑
\(0=\langle T^*T^{k-1}v,T^{k-2}v\rangle=\|T^{k-1}v\|^2\),
这又推出\(T^{k-1}v=0\). 依次递推可知\(Tv=0\),
故证得\(\Null T^k= \Null T\).

另一方面,\(\range T^k\subset \range T\)是显然的,而\(\dim\range T^k=\dim V-\dim \Null T^k=\dim V-\dim \Null T=\dim\range T\),
所以\(\range T^k=\range T\).

\bigskip

\noindent{} 7. 证明没有自伴算子\(T\in\calL(\R^3)\)使得
\[T(1,2,3)=(0,0,0),T(2,5,7)=(2,5,7).\]

\smallskip

\noindent{} 解:假设存在这样的自伴算子,那么
\[\langle T(1,2,3),(2,5,7)\rangle =\langle (1,2,3),T(2,5,7)\rangle.\]
左侧等于\(0\), 右侧等于\(\langle (1,2,3),(2,5,7)\rangle \ne0\),
显然不可能成立,故不存在这样的自伴算子。

\bigskip

\section{谱定理及其应用}\label{ux8c31ux5b9aux7406ux53caux5176ux5e94ux7528}

\noindent{} 1.
证明:在复内积空间上,一个正规算子是自伴的当且仅当它的所有特征值都是实的。

\smallskip

\noindent{}
解:如果一个正规算子是自伴算子,那么它所有的特征值显然都是实的,因为自伴算子特征值都是实的。反之,设\(T\)是复内积空间上的一个正规算子,它的特征值都是实的。由谱定理,可以找到一组规范正交基使得\(\calM(T)\)是对角阵,对角元素是它的特征值。在规范正交基下,\(\calM(T^*)=\calM(T)^*=\calM(T)\),
这就说明了\(T\)是自伴的。

\bigskip

\noindent{} 2.
设\(V\)是复内积空间,\(T\in\calL(V)\)是正规算子使得\(T^9=T^8\).
证明\(T\)是自伴的,并且\(T^2=T\).

\smallskip

\noindent{}
解:由谱定理,复内积空间\(V\)上的正规算子\(T\)可以对角化,即可以找到一组规范正交基,使得\(\calM(T)\)是对角矩阵,设其对角元素分别为\(\lambda_1,\lambda_2,\cdots,\lambda_n\)。那么在这组基下,\(\calM(T^9)=\calM(T)^9,\calM(T^8)=\calM(T)^8\),
所以\(\lambda_i^9=\lambda_i^8\), 这说明\(\lambda_i=0\)或\(\lambda_i=1\).
显然这些特征值都是实的,由上一题可知\(T\)是自伴的,且\(\lambda_i^2=\lambda_i\)对所有\(i=1,2,\cdots,n\)都成立,故\(T^2=T\).

\bigskip

\noindent{} 3.
设\(V\)是复内积空间,证明\(V\)上的每个正规算子都有平方根。(算子\(S\in\calL(V)\)称为\(T\in\calL(V)\)的平方根如果\(S^2=T\).)

\smallskip

\noindent{}
解:设\(T\in\calL(V)\)是一个自伴算子,由谱定理,可以找到\(V\)的一组规范正交基,\(T\)在这组基下的矩阵是一个对角阵,设其对角元为\(\lambda_1,\cdots,\lambda_n\)。定义\(S\),它在这组基下的矩阵为对角阵且对角元为\(\sqrt{\lambda_1},\cdots,\sqrt{\lambda_n}\),
那么显然有\(S^2=T\).

\smallskip

\noindent{} Remark:
由于考虑的是复内积空间,我们可以将特征值开方,得到复数也没有关系。如果是实内积空间显然不能这么做。实际上这个结论在实内积空间上的版本更为常用:实内积空间上的任何正自伴算子一定存在平方根。

\bigskip

\noindent{} 4. 给出实内积空间\(V\),
算子\(T\in\calL(V)\)以及满足\(\alpha^2<4\beta\)的实数\(\alpha,\beta\)使得\(T^2+\alpha T+\beta I\)不可逆。

\smallskip

\noindent{} 解:令\(V=\R^2\), \(T(x,y)=(-y,x)\), \(\alpha=0,\beta=1\).
那么\(T^2=-T\), \(T^2+\alpha T+\beta I=0\)显然不可逆。

\noindent{} 5.
证明或举反例:\(V\)上每个自伴算子都有立方根。(算子\(S\in\calL(V)\)称为\(T\in\calL(V)\)的立方根,如果\(S^3=T\).)

\smallskip

\noindent{} 解:
设\(T\in\calL(V)\)是一个自伴算子,由谱定理,可以找到\(V\)的一组规范正交基,\(T\)在这组基下的矩阵是一个对角阵,设其对角元为\(\lambda_1,\cdots,\lambda_n\)。定义\(T\),它在这组基下的矩阵为对角阵且对角元为\(\lambda_1^{\frac13},\cdots,\lambda_n^{\frac13}\),
那么显然有\(S^3=T\).

\smallskip

\noindent{} Remark:
显然这题的结论是可以推广的,任何奇次方根都存在。对于偶次方根,则还需要算子是正的。

\bigskip

\noindent{} 6.
设\(T\in\calL(V)\)是自伴的,\(\lambda\in\bfF,\varepsilon>0\).
证明,若有\(v\in V\)使得\(\|v\|=1\),
并且\(\|Tv-\lambda v\|<\varepsilon\), 则有\(T\)
的特征值\(\lambda'\)使得\(|\lambda-\lambda'|<\varepsilon\).

\smallskip

\noindent{}
解:由谱定理,\(T\)可以在某组规范正交基\((e_1,e_2,\cdots,e_n)\)下写成对角阵,对角元即为特征值\(\lambda_1,\lambda_2,\cdots,\lambda_n, Te_j=\lambda_j e_j\).
利用反证法,假设\(|\lambda_j-\lambda|\ge\varepsilon\).
设\(v=\sum_{i=1}^nx_ie_i\),
则\(Tv-\lambda v=\sum_{i=1}^nx_i Te_i-\lambda\sum_{i=1}^nx_ie_i\),
\[\|Tv-\lambda v\|^2=\|\sum_{i=1}^n(\lambda_i-\lambda)x_ie_i\|^2=\sum_{i=1}^n(\lambda_i-\lambda)^2x_i^2\ge\varepsilon^2\|v\|^2=\varepsilon^2.\]
这与题目的条件矛盾。所以\(T\)一定有某个特征值\(\lambda'\)使得\(|\lambda-\lambda'|<\varepsilon\).

\bigskip

\section{正算子}\label{ux6b63ux7b97ux5b50}

\noindent{} 1. 证明\(V\)上两个正算子的和是正的。

\smallskip

\noindent{}
解:设\(S,T\)是\(V\)上的两个正算子,那么\(\langle (S+T)v,v\rangle = \langle Sv,v\rangle+\langle Tv,v\rangle\ge 0,\forall v\in V\),
并且\(S+T\)也是自伴的,故\(S+T\)也是正算子。

\bigskip

\noindent{} 2.
证明:若\(T\in\calL(V)\)是正的,则对于每个正整数\(k\),\(T^k\)都是正的。

\smallskip

\noindent{}
解:我们可以对\(k\)分奇数和偶数的情况进行讨论。如果\(k\)是偶数,那么\(k=2m\),\(\langle T^kv,v\rangle = \langle T^{2m}v,v\rangle =\langle T^mv,T^mv\rangle\ge 0\).
如果\(k\)是奇数,那么\(k=2m+1\),
\(\langle T^kv,v\rangle=\langle T^{2m+1}v,v\rangle = \langle T(T^mv),T^mv\rangle\ge 0\).
因此对每个正整数\(k\), \(T^k\)都是正的。

\bigskip

\noindent{} 3.
设\(T\)是\(V\)上的正算子。证明\(T\)可逆当且仅当对每个\(v\in V-\{0\}\)都有\(\langle Tv,v\rangle>0\).

\smallskip

\noindent{}
解:设\(T\)可逆。由于\(T\)是正算子,存在\(S\in\calL(V)\)使得\(T=S^*S\).
对于任意\(v\in V-\{0\}\), \(Sv\ne 0\), 否则\(Tv= S*Sv=0\),
这与\(T\)是正算子矛盾。那么,\(\langle Tv,v\rangle = \langle S^*Sv,v\rangle=\langle Sv,Sv\rangle\ge 0\).

反之,若\(\langle Tv,v\rangle>0,\forall v\in V-\{0\}\),
那么\(\Null T=\{0\}\), 显然说明\(T\)可逆。

\end{document}
