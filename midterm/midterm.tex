\documentclass[12pt]{article}
\textheight 23.5cm \textwidth 15.8cm
%\leftskip -1cm
\topmargin -1.5cm \oddsidemargin 0.3cm \evensidemargin -0.3cm
%\documentclass[final]{siamltex}

\usepackage{verbatim}
\usepackage{fancyhdr}
\usepackage{graphicx}
\usepackage{amssymb}
\usepackage{mathrsfs}
\usepackage{amsmath}
\usepackage{CTeX}
\usepackage{enumerate}   



\title{\bf 线性代数期中考试参考题}
\author{李弢}
\date{}

\begin{document}
\maketitle
\noindent{} Q1. (线性空间)
\begin{enumerate}[(a)]
\item 给定集合$\mathbb{R}_{++}=\{x>0|x\in\mathbb{R}\}$ 及其上面的加法运算与数乘运算:
\[x+y = xy\quad, \quad\lambda x= x^{\lambda}.\]
问$\mathbb{R}_{++}$在上述加法与数乘的意义下是否构成一线性空间?若不是,请说明理由;若是,请验证,并求出维数,并进一步给出它到$\mathbb{R}^{\mathrm{dim}\, \mathbb{R}_{++}}$的一个可逆线性映射。
\item 任给一线性空间$V$,我们把$\mathcal{L}(V;\mathbb{F})$称为$V$的对偶空间,可记作$V^*$. 设$V$是有限维的,$\{v_1,v_2,\cdots,v_n\}$是一组基,定义$v_i^*\in V^*$,
\[v_i^*:V\rightarrow \mathbb{F}, v_i^*(v_j) = \delta_{ij},\quad\forall i,j=1,2,\cdots,n,\]
其中$\delta_{ij}$当且仅当$i=j$时为1,否则为0. 求证$\{v_1^*,v_2^*,\cdots,v_n^*\}$是$V^*$的一组基。
\item 对于$V$的对偶空间$V^*$,我们还可以考虑它的对偶空间$V^{**}$. 证明,$V$是$V^{**}$子空间。特别地,当$V$为有限维空间时,$V=V^{**}$. 
\end{enumerate}

\noindent{} Solution.
\begin{enumerate}[(a)]
\item 根据定义可以验证这是一个线性空间,其加法单位元为$1$,维数为1。对数映射$\log:\mathbb{R}_{++}\rightarrow \mathbb{R}$ 是一可逆线性映射:
\[\log(\lambda x+\mu y) =\log(x^{\lambda}y^{\mu})=\lambda \log(x)+\mu\log(y).\]
其逆为$\exp: \mathbb{R}\rightarrow\mathbb{R}_{++}$.

\item 首先注意到若$v=\sum_{j=1}^n\lambda_j v_j$,那么
\[v_i^*(v) = v_i^*(\sum_{j=1}^n\lambda_j v_j)=\sum_{j=1}^n\lambda_j v_i^*(v_j)=\lambda_i.\]
我们先说明$\{v_1^*,v_2^*,\cdots,v_n^*\}$是$V^*$的张成组。任意$l \in V^*, v\in V$,有
\[l(v) = l(\sum_{i=1}^n \lambda_i v_i) = \sum_{i=1}^n\lambda_i l (v_i) =  \sum_{i=1}^n l (v_i) v_i^*(v),\]
这就说明了任意$l \in V^*$都可以由$\{v_1^*,v_2^*,\cdots,v_n^*\}$线性表示。

再来说明$\{v_1^*,v_2^*,\cdots,v_n^*\}$是线性无关的。设存在$\lambda_1,\lambda_2,\cdots,\lambda_n$使得$\sum_{i=1}^n\lambda_i v_i^*=0$,那么作用在$v_j,j=1,2,\cdots,n$上可知$\lambda_j=0$,进而说明了它们线性无关。

因此$\{v_1^*,v_2^*,\cdots,v_n^*\}$ 是$V^*$的一组基。特别地,若$V$是有限维线性空间,$V^*$也是有限维线性空间,且维数与$V$相等。

\item 任取$v\in V$,可以将其视为$V^*$上的线性函数:$v(l) = l(v)$. 这样定义的函数确实是线性的:
\[v(\lambda l_1+\mu l_2) = (\lambda l_1+\mu l_2)(v) = \lambda l_1(v)+\mu l_2(v)=\lambda v(l_1)+\mu v(l_2).\] 
这就说明了$V$是$V^{**}$的子空间。若$V$是有限维的,那么由$(b)$可知,$\mathrm{dim}\, V^* = \mathrm{dim}\, V$,再一次利用(b)可知$\mathrm{dim}\, V^{**}=\mathrm{dim}\, V^* = \mathrm{dim}\,V$, 由此可知$V^{**}=V$.
\end{enumerate} 

\bigskip

\noindent{} Q2. (维数公式)
\begin{enumerate}[(a)]
\item 设$V$是有限维线性空间,$U_1,U_2$是$V$的子空间,那么有怎样的维数公式(不用证明)?
\item 设$V,W$是有限维线性空间,$T\in\mathcal{L}(V;W)$是从$V$到$W$的线性映射,记T的像与核分别为$\mathrm{range}\,T$与$\mathrm{null}\,T$,那么有怎样的维数公式(不用证明)?
\item 给定两个线性空间$U_1,U_2$,我们定义它们的笛卡尔积$U_1\times U_2=\{(u_1,u_2),u_1\in U_1,u_2\in U_2\}$,定义加法运算和数乘运算为
\[(u_1,u_2)+(u_1',u_2')=(u_1+u_1',u_2+u_2')\quad,\quad \lambda(u_1,u_2)=(\lambda u_1,\lambda u_2).\]
请验证$U_1\times U_2$是一个线性空间。若$U_1,U_2$均是有限维,再求$U_1\times U_2$的维数。
\item 设$V$是有限维线性空间,$U_1,U_2$是$V$的子空间,考虑$T:U_1\times U_2\rightarrow V, T((u_1,u_2))=u_1+u_2$. 证明$T\in\mathcal{L}(U_1\times U_2;V)$,并利用该$T$和(b)对(a)中的结果给出证明。
\item 考虑一个齐次线性方程组
\[\left\{
\begin{aligned}
&a_{11}x_1 + &a_{12}x_2 +\cdots+&a_{1n}x_n =& 0\\
&a_{21}x_1 + &a_{22}x_2 +\cdots+&a_{2n}x_n = &0\\
&\cdots&\cdots&&\cdots\\
&a_{m1}x_1 + &a_{m2}x_2 +\cdots+&a_{mn}x_n = &0\\
\end{aligned}\right.
\]
记该线性方程组的系数矩阵为$A\in\mathbb{R}^{m\times n}, [A]_{ij}=a_{ij}$,$A$的第$j$列为$A_j,j=1,2,\cdots,n$,解空间为$U$,则$\mathrm{dim}\, U$与系数矩阵$A$有何关系?

\end{enumerate}

\noindent{} Solution. 
\begin{enumerate}[(a)]
\item $\mathrm{dim}\, U_1+U_2 = \mathrm{dim}\, U_1+\mathrm{dim}\, U_2 - \mathrm{dim}\, U_1\cap U_2. $
\item $\mathrm{dim}\,V = \mathrm{dim}\, \mathrm{range}\,T+\mathrm{dim}\,\mathrm{null}\, T.$
\item 逐一验证线性空间的性质即可。取$U_1$一组基$\{e_1,e_2,\cdots, e_{\mathrm{dim}\,U_1}\}$,$U_2$一组基$\{f_1,f_2,\cdots,f_{\mathrm{dim}\,U_2}\}$,那么说明
$$\{(e_1,0),(e_2,0),\cdots, (e_{\mathrm{dim}\,U_1},0)\}\cup\{(0,f_1),(0,f_2),\cdots,(0,f_{\mathrm{dim}\,U_2})\}$$
是$U_1\times U_2$的一组基即可知,
\[\mathrm{dim}\,U_1\times U_2 = \mathrm{dim}\, U_1+\mathrm{dim}\,U_2. \]
\item 容易验证$T$是线性映射,下面考虑其像与核:
\[\mathrm{range}\,T = \{u_1+u_2|u_1\in U_1,u_2\in U_2\} = U_1+U_2,\]
\[\mathrm{null}\,T =\{u_1+u_2 =0|u_1\in U_1,u_2\in U_2\}.\]
实际上,$\mathrm{null}\,T =\{(u,-u)|u\in U_1\cap U_2\}$. 这是因为任意$(u_1,u_2)\in \mathrm{null}\,T$, 都有$-u_1\in U_2$,从而$u_1\in U_2$,因此可以写成$(u_1,-u_1)$,其中$u_1\in U_1\cap U_2$. 另一方面,对于任意$u\in U_1\cap U_2$,$T((u,-u))=0$. 由(b)可知,
\[\mathrm{dim}\,U_1 +\mathrm{dim}\,U_2
=\mathrm{dim}\,(U_1\times U_2) 
= \mathrm{dim}\,\mathrm{range}\,T+ \mathrm{dim}\,\mathrm{null}\,T 
= \mathrm{dim}\,U_1+\mathrm{dim}\,U_2 +\mathrm{dim}\,U_1\cap U_2.
\]
\item 考虑线性映射$T:\mathbb{F}^n\rightarrow \mathbf{F}^m, T(x) = Ax$, 那么
\[\mathrm{null}\, T=U\quad ,\quad \mathrm{range}\,T = \mathrm{span}\,\{A_1,A_2,\cdots,A_n\}.\]
由(b)可知,
\[\mathrm{dim}\,U=\mathrm{dim}\,\mathrm{null}\,T = m-\mathrm{dim}\,\mathrm{span}\,\{A_1,A_2,\cdots,A_n\}.\]
\end{enumerate}

\bigskip

\noindent{} Q3 (投影算子) 设$V$是有限维线性空间,$P\in\mathcal{L}(V)$,如果$P$满足$P^2 = P$,那么我们称其为投影算子。
\begin{enumerate}[(a)]
\item 设$V=U\oplus W$,证明$P_{U,W}$是投影算子。
\item 证明:若$P\in\mathcal{L}(V)$是一投影算子,$\lambda$是$P$的特征值,则$\lambda=0$或$1$.
\item 证明:若$P\in\mathcal{L}(V)$是一投影算子,则$P$可对角化。
\item 证明$V$上所有的投影算子都可以写成(a)中的形式。
\end{enumerate}

\noindent{} Solution
\begin{enumerate}[(a)]
\item $P_{U,W}^2 = P_{U,W}$课上已经证明过。

\item 设$\lambda$为$P$的特征值,$v$是对应的(非零)特征向量,那么$Pv=\lambda v$. 由于$P$是投影算子,两边同时用$P$作用,得到$Pv = P^2v=\lambda Pv$,即$(1-\lambda) Pv=0$. 由某道作业题,可知$1-\lambda=0$或$Pv=0$. 如果$1-\lambda=0$,就是$\lambda=1$; 如果$Pv=0$,由于$v$是对应于$\lambda$的特征向量,可知$\lambda=0$.

\item 由于$P$是投影算子,$P(P-I)=0$,也就是说$\mathrm{range}\, (P-I) \subset \mathrm{null}\, P$. 由维数公式,
\[\mathrm{dim}\, V = \mathrm{dim}\, \mathrm{range}\,(P-I)+\mathrm{dim}\,\mathrm{null}\,(P-I)\le \mathrm{dim}\, \mathrm{null}\,P+\mathrm{dim}\,\mathrm{null}\,(P-I).\]
若$P$只有0特征值或1特征值,由上式可以看出$\mathrm{null}\,P=V$或$\mathrm{null}\,(P-I)=V$,$P$显然可以对角化。否则,由于0和1 都是$P$的特征值,$\mathrm{null}\,P \oplus \mathrm{null}\,(P-I)$是$V$的子空间,故
\[\mathrm{dim}\, \mathrm{null}\,P+\mathrm{dim}\,\mathrm{null}\,(P-I)\le\mathrm{dim}\, V.\]
结合上两式可知
\[\mathrm{dim}\, \mathrm{null}\,P+\mathrm{dim}\,\mathrm{null}\,(P-I)=\mathrm{dim}\, V.\]
这就说明了$P$可以对角化。

\item 由(c)中的结论,$V =  \mathrm{null}\,(P-I)\oplus  \mathrm{null}\,P$. 对于任意$v\in V, v= v_1+v_2, v_1\in  \mathrm{null}\,(P-I), v_2 \in  \mathrm{null}\,P$,那么$P(v)=P(v_1)+P(v_2)=v_1$. 这就说明了$P = P_{ \mathrm{null}\,(P-I), \mathrm{null}\,P}$.
\end{enumerate}

\end{document}